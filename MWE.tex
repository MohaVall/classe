\documentclass[twoside,12pt]{Classes/aesm_edspia1}

\usepackage[latin1]{inputenc}
\usepackage[english,french]{babel}
\usepackage[T1]{fontenc}
\usepackage{amsmath}
\usepackage{lmodern}%font modern
\rmfamily
\DeclareFontShape{T1}{lmr}{bx}{sc}{<->ssub * cmr/bx/sc}{} %manque une police en lmodern
\usepackage{lettrine}
\usepackage{tabularx}
\usepackage{epsfig, floatflt, amssymb}
%\usepackage{wrapfig}%figure entour� de texte.
\usepackage{moreverb} %% pour le verbatim en boite
\usepackage{cases}%equations en systemes num�rot�s - soluce possible package : CASES
%\usepackage{slashbox} %% pour couper les colonnes des tableaux en diagonale
%\usepackage{layout}
%\usepackage{showkeys} %% pour voir les labels
\usepackage{multirow} %% pour regrouper un texte sur plusieurs lignes dans une table
\usepackage{url} %% pour citer les url par \url
\usepackage[all]{xy} %% pour la barre au dessus des symboles
%\usepackage{shorttoc} %% pour plusieurs tables des mati�res par la commande \shorttableofcontents{Titre}{profondeur}.
\usepackage{textcomp} %% pour le symbol pour mille par \textperthousand et degr�s par \degres
\usepackage[right]{eurosym}
\usepackage{setspace} %interligne simple, double etc...
\usepackage{eurosans} %%pour le symbole \euro
\usepackage{epic,eepic}
\usepackage{soul}
%\usepackage{lineno}%num�roter les lignes
\usepackage[nottoc]{tocbibind} % tables des figures, des matieres et autres dans la TOC
%\usepackage{palatino}
\usepackage{fancybox}
\usepackage[leftcaption]{sidecap}
%\usepackage[labelsep=endash, textfont={normalsize,onehalfspacing}, margin=5pt, format=hang, labelfont=bf]{caption}
\usepackage[labelsep=endash, textfont={footnotesize, singlespacing}, margin=10pt, format=plain, labelfont=bf]{caption}
\usepackage[Conny]{fncychap} %en tete chapitrage
\newcommand{\ie}{c.-\`a-d.~}
\hbadness=10000% pb d'overfull box r�gl�
\hfuzz=50pt
\pdfcompresslevel9 % pour compresser le pdf final au maximum
\pdfoptionpdfminorversion=5 % pour accept� les images PDF version 1.5 (ex: celles produites par Office 2007)
\def\underscore{\char`\_}
\makeatletter
\renewcommand{\thesection}{\arabic {section}}
\renewcommand{\SC@figure@vpos}{c}% centrer verticalement le caption avec le package sidecap...
\renewcommand{\fnum@figure}{\small\textbf{Figure~\thefigure}}
\renewcommand{\fnum@table}{\small\textbf{Tableau~\thetable}}
%\newcommand\figcaption{\def\@captype{figure}\caption}
%\newcommand\tabcaption{\def\@captype{table}\caption}
\makeatother
\usepackage{subfig}
\def\thechapter{\Roman{chapter}}
\def\thechapter{\Roman{chapter}}
\usepackage{float}
\usepackage{here}
%\newlength\longest
\usepackage{pifont}
\input{tcilatex}


%%%%%%%%%%%%%%%%%%%%%%%%%%%%%%%%%%%%%%%%%%%
\begin{document}
%%%%%%%%%%%%%%%%%%%%%%%%%%%%%%%%%%%%%%%%%%%
\renewcommand\figurename{\small\textbf{Figure}}

\addtocounter{page}{-1}%pour revenir � 0



\makethese %% cr�e la couverture.

\onehalfspacing


\fancyhead[LE,RO]{Introduction }
%\fancyhead[RO]{\bfseries\rightmark}
%\fancyhead[LE]{\bfseries\leftmark}
\fancyfoot[RO]{\thepage}
\fancyfoot[LE]{\thepage}
\renewcommand{\headrulewidth}{0.5pt}
\renewcommand{\footrulewidth}{0pt}
%\renewcommand{\chaptermark}[1]{\markboth{\MakeUppercase{\chaptername~\thechapter. #1 }}{}}
%\renewcommand{\sectionmark}[1]{\markright{\thechapter.\thesection~ #1}}
%%%%%%%%%%%%%%%%%%%%%%%%%%%%%%%%%%%%%%%%%%%%
\chapter*{Introduction g\'en\'erale}

\addcontentsline{toc}{chapter}{Introduction g�n�rale}
%==================================================================================================%
Dans ce chapitre, nous allons parler des
g\'{e}n\'{e}ralit\'{e}s sur la
SUSY. Nous commencerons par la description de l'alg\`{e}bre supersym\'{e}%
trique, ensuite nous allons donner ses repr\'{e}sentations. Et
finalement on va parler des superespaces et
superchamps.

\section{Mod\`{e}le particulier avec une vari\'{e}t\'{e} hyper-K\"{a}%
hlerienne et un groupe de jauge $U(1)\times U(1)$}

Dans les parties pr\'{e}c\'{e}dentes, on a vu que ce mod\`{e}le
contient un multiplet de mati\`{e}re d\'{e}crit par un multiplet
tensoriel ou un hyper-multiplet, et $n$ superchamps de Maxwell
$U(1)$. Dor\'{e}navant on va
le restreindre par l'imposition du potentiel de K\"{a}hler suivant:%
\begin{equation}
K\left( Q^{u},\overline{Q}^{\overline{u}}\right) =\overline{Q}^{1}Q^{1}+%
\overline{Q}^{2}Q^{2}=\left( \overline{Q}^{1}\text{ }Q^{1}\right)
\begin{pmatrix}
1 & 0 \\
0 & 1%
\end{pmatrix}%
\binom{Q^{1}}{Q^{2}}.  \label{20}
\end{equation}%
On remarque que la m\'{e}trique de K\"{a}hler est Ricci-plate, on a
aussi
chang\'{e} les notations de deux multiplet chiral de l'hypermultiplet: $%
\left( \overline{Q}^{1},\text{ }Q^{1}\right) .$ le mod\`{e}le dual \`{a} (\ref%
{17}) avec le potentiel de K\"{a}hler particulier (\ref{20}):%
\begin{eqnarray}
\mathcal{L} &\mathcal{=}&\mathcal{L}_{gauge}+\dint d^{4}\theta
\left[
\overline{Q}^{1}e^{-2g_{a}V^{a}}Q^{1}+\overline{Q}^{2}e^{2g_{a}V^{a}}Q^{2}%
\right]  \label{21} \notag \\
&&+\dint d^{2}\theta \left[ \left( m+\sqrt{2}ig_{a}X^{a}\right) Q^{1}Q^{2}-%
\frac{ig_{a}}{k_{a}}Y\right] , \label{21}
\end{eqnarray}%
o\`{u} $m$ est un param\`{e}tre de masse, et $Q^{1}$ et $Q^{2}$ ont
deux charges de $U(1)$ opos\'{e}es.\newline Pour determiner $H\left(
L,\Phi ,\overline{\Phi }\right) $, on doit effectuer une
transformation de dualit\'{e} inverse, pour ceci on doit faire
le changement de variables suivant:%
\begin{equation}
Q^{1}=\sqrt{\frac{\Phi }{\sqrt{2}}}e^{-\Phi \prime
},Q^{2}=i\sqrt{\frac{\Phi
}{\sqrt{2}}}e^{\Phi \prime }\  \  \Rightarrow Q^{1}Q^{2}=\frac{i}{\sqrt{2}}%
\Phi .  \label{tfn}
\end{equation}%
Alors (\ref{21}) devient:%
\begin{eqnarray*}
\mathcal{L} &\mathcal{=}&\mathcal{L}_{gauge}+\frac{1}{\sqrt{2}}\dint
d^{4}\theta \sqrt{\overline{\Phi }\Phi }\left[ e^{\overline{\Phi
}\prime +\Phi \prime +2g_{a}V^{a}}+e^{-\overline{\Phi }\prime -\Phi
\prime
-2g_{a}V^{a}}\right] \\
&&+\dint d^{2}\theta \left( m+\sqrt{2}ig_{a}X^{a}\right) \frac{i}{\sqrt{2}}%
\Phi -\frac{ig_{a}}{k_{a}}Y+h.c.
\end{eqnarray*}%
Le potentiel de K\"{a}hler depend maintenant de $\overline{\Phi
}\prime
+\Phi \prime ,$ alors le model dual \`{a} (\ref{21}) est donn\'{e} par:%
\begin{eqnarray}
\mathcal{L} &\mathcal{=}&\mathcal{L}_{gauge}+\dint d^{4}\theta H\left( L,\Phi ,%
\overline{\Phi }\right) +\dint d^{2}\theta \left( m+\sqrt{2}%
ig_{a}X^{a}\right) \frac{i}{\sqrt{2}}\Phi  \label{24} \\
&&-\frac{ig_{a}}{k_{a}}Y+h.c,  \notag
\end{eqnarray}%
avec:%
\begin{equation}
H\left( L,\Phi ,\overline{\Phi }\right) =\sqrt{L^{2}+2\Phi \overline{\Phi }}%
-L\ln \left( L+\sqrt{L^{2}+2\Phi \overline{\Phi }}\right)
+2g_{a}LV^{a}. \label{25}
\end{equation}%
En effet:%
\begin{equation}
\dint d^{4}\theta \sqrt{\frac{\overline{\Phi }\Phi }{2}}\left[ e^{\overline{%
\Phi }\prime +\Phi \prime +2g_{a}V^{a}}+e^{-\overline{\Phi }\prime
-\Phi \prime -2g_{a}V^{a}}\right] ,  \label{dl1}
\end{equation}%
est \'{e}quivalant \`{a}\footnote{%
En \'{e}limiant L dans (\ref{dl2}), par son \'{e}quation du
mouvement, on trouve facilement (\ref{dl1})}:
\begin{equation}
\dint d^{4}\theta \left[ \sqrt{\frac{\overline{\Phi }\Phi
}{2}}\left[ e^{A}+e^{-A}\right] -L\left( A-2g_{a}V^{a}\right)
\right] ,  \label{dl2}
\end{equation}%
o\`{u} $A$ est un superchamp arbitraire r\'{e}el et $L$ est un superchamp lin%
\'{e}aire qui joue le r\^{o}le d'un multiplicateur de Lagrange.
L'\'{e}quation du mouvement de A implique:%
\[
\sqrt{\frac{\overline{\Phi }\Phi }{2}}\left( e^{A}-e^{-A}\right)
-L=0,
\]%
dont la solution est:%
\begin{equation}
e^{A}=\frac{L}{\sqrt{2\left\vert \Phi \right\vert ^{2}}}\left( L+\sqrt{%
L^{2}+2\left\vert \Phi \right\vert ^{2}}\right) .  \label{dl3}
\end{equation}%
En substituant (\ref{dl3}) dans (\ref{dl2}) on trouve:%
\[
\dint d^{4}\theta \left[ \sqrt{L^{2}+2\left\vert \Phi \right\vert
^{2}}-L\ln
\left( L+\sqrt{L^{2}+2\left\vert \Phi \right\vert ^{2}}\right) +2Lg_{a}V^{a}%
\right]+un\,terme\,de\,surface =\dint d^{4}\theta H\left( L\right).
\]%
\ Il est maintenant simple de v\'{e}rifier que $H\left( L,\Phi ,\overline{%
\Phi }\right) $ satisfait l'\'{e}quation de Laplace. Le lagrangien
(\ref{21}) depend d'un champ 4-forme $Y$ mais son \'{e}quation de
mouvement pose la
contrainte suivante:%
\begin{equation}
\frac{g_{a}}{k_{a}}=0.  \label{26}
\end{equation}%
Alors la d\'{e}pendance du champ non dynamique $Y$ est
\'{e}limin\'{e}e grace \`{a} la contrainte (\ref{26}).\newline
Remarque: dans le cas d'un seul $U(1)$ la contrainte (\ref{26}) devient: $%
\frac{g}{k}=0$, alors soit l'hypermultiplet est non charg\'{e} sous
$U(1)$ $\left( g=0\right) $, ou la seconde SUSY est r\'{e}alis\'{e}e
lin\'{e}airement $\left( \frac{1}{k}=0\right) $, et
dans les deux cas on peut pas bris\'{e}e la SUSY $\mathcal{N}=2$ \`{a} deux \'{e}chelles diff%
\'{e}rentes.

\section{Vide du mod\`{e}le}

Comme ce mod\`{e}le contient un hypermultiplet $\left( 0^{4},\frac{1}{2}%
^{2}\right) $ et un multiplet de Maxwell $\left( 0^{2},\frac{1}{2}%
^{2},1\right) $ alors on a deux phases; la phase de Higgs qui est associ\'{e}%
e \`{a} l'hypermultiplet et dans laquelle le groupe de jauge
$U(1)\times U(1) $ est intact \cite{1}, et celle de Coulomb qui est
associ\'{e}e \`{a} le multiplet de Maxwell et dans laquelle le
groupe de jauge $U(1)\times U(1)$ est bris\'{e}. on remarque qu'on
ne peut utiliser dans la phase de Coulomb que le formalisme
d'hypermultiplet, car si $U(1)\times U(1)$ est intact
alors la valeur moyenne dans le vide ( VEV\ ) des champs scalaires $%
\left \langle q^{u}\right \rangle \ $est nulle, mais la
d\'{e}finition des champs (\ref{tfn}) montre que $\left \langle
q^{u}\right \rangle $ ne peut pas \^{e}tre nul, par cons\'{e}quent
on peut utiliser le formalisme tensoriel seulement dans la phase de
Higgs, mais l'analyse de cette branche est plus simple si on utilise
le formalisme tensoriel.

\subsection{Phase de Coulomb}

Comme ce qu'on a vu, dans le formalisme d'hypermultiplet le
lagrangien est
donn\'{e} par (\ref{21}) \cite{1}:%
\begin{eqnarray*}
\mathcal{L} &\mathcal{=}&\dint d^{4}\theta \left( \overline{Q}%
^{1}e^{-2g_{a}V^{a}}Q^{1}+\overline{Q}^{2}e^{2g_{a}V^{a}}Q^{2}\right)
+\dint
d^{4}\theta \xi _{a}V^{a} \\
&&+\frac{i}{2}\dint d^{4}\theta \left[
\overline{\mathcal{F}}_{a}\left(
\overline{X}^{b}\right) X^{a}-\mathcal{F}_{a}\left( X^{b}\right) \overline{X}%
^{a}\right] \\
&&+\dint d^{2}\theta \left( m+\sqrt{2}ig_{a}X^{a}\right) Q^{1}Q^{2}+h.c-%
\frac{i}{4}\dint d^{2}\theta \mathcal{F}_{ab}\left( X^{c}\right)
W^{a}W^{b}+h.c \\
&&-\dint d^{2}\theta \left( \frac{e}{4}X^{a}+\frac{i}{4k_{a}}\mathcal{F}%
_{a}\left( X^{b}\right) \right) +h.c.
\end{eqnarray*}%
Le potentiel scalaire est donn\'{e} par \cite{WB}:%
\begin{equation*}
V_{s}=\overline{F}^{\overline{q}^{1}}F^{q^{1}}+\overline{F}^{\overline{q}%
^{2}}F^{q^{2}}+h^{ab}\overline{F}^{\overline{x}^{a}}F^{x^{b}}+\frac{1}{2}%
h^{ab}D_{a}D_{b}.
\end{equation*}%
Calculons les expressions de $\overline{F}^{\overline{q}^{1}}$et $\overline{F%
}^{\overline{q}^{2}}$:%
\begin{equation*}
\begin{tabular}{l}
$\dint d^{4}\theta \left( \overline{Q}^{\overline{1}}e^{-2g_{a}V^{a}}Q^{1}%
\right) +\dint d^{2}\theta \left( m+\sqrt{2}ig_{a}X^{a}\right)
Q^{1}Q^{2}+h.c $ \\
$=\dint d^{4}\theta \left[ \overline{Q}^{\overline{1}}\left(
1-2g_{a}V^{a}+...\right) Q^{1}\right] +\dint d^{2}\theta \left( m+\sqrt{2}%
ig_{a}X^{a}\right) Q^{1}Q^{2}+h.c,$ \\
$\supset \dint d^{4}\theta \left[ \left( \overline{\theta }^{2}\overline{F}^{%
\overline{q}^{1}}\right) \left( 1\right) \left( \theta ^{2}F^{q^{1}}\right) %
\right] +\dint d^{2}\theta \left( m+\sqrt{2}ig_{a}x^{a}\right)
\left( \theta
^{2}F^{q^{1}}\right) q^{2}+h.c,$ \\
$=\overline{F}^{\overline{q}^{1}}F^{q^{1}}+\left( m+\sqrt{2}%
ig_{a}x^{a}\right) F^{q^{1}}q^{2}+h.c.$%
\end{tabular}%
\end{equation*}%
Alors l'\'{e}quation de mouvement de $F^{q^{1}}$:%
\begin{equation*}
\overline{F}^{\overline{q}^{1}}+\left( m+\sqrt{2}ig_{a}x^{a}\right)
q^{2}=0\Rightarrow \overline{F}^{q^{1}}=-\left(
m+\sqrt{2}ig_{a}x^{a}\right) q^{2}=-m_{eff}q^{2},
\end{equation*}%
avec%
\begin{equation*}
m_{eff}=m+\sqrt{2}ig_{a}x^{a}.
\end{equation*}

\begin{itemize}
\item
\begin{eqnarray*}
&&\dint d^{4}\theta \left(
\overline{Q}^{2}e^{2g_{a}V^{a}}Q^{2}\right)
+\dint d^{2}\theta \left( m+\sqrt{2}ig_{a}X^{a}\right) Q^{1}Q^{2}+h.c \\
&=&\dint d^{4}\theta \left[ \overline{Q}^{2}\left(
1+2g_{a}V^{a}+...\right) Q^{2}\right] +\dint d^{2}\theta \left(
m+\sqrt{2}ig_{a}X^{a}\right)
Q^{1}Q^{2}+h.c, \\
&\supset &\dint d^{4}\theta \left[ \left( \overline{\theta }^{2}\overline{F}%
^{\overline{q}^{2}}\right) \left( 1\right) \left( \theta
^{2}F^{q^{2}}\right) \right] +\dint d^{2}\theta \left( m+\sqrt{2}%
ig_{a}x^{a}\right) \left( \theta ^{2}F^{q^{2}}\right) q^{1}+h.c, \\
&=&\overline{F}^{\overline{q}^{2}}F^{q^{2}}+\left( m+\sqrt{2}%
ig_{a}x^{a}\right) F^{q^{2}}q^{1}+h.c.
\end{eqnarray*}%
Alors l'\'{e}quation de mouvement de $F^{q^{2}}$:%
\begin{equation*}
\overline{F}^{\overline{q}^{2}}+\left( m+\sqrt{2}ig_{a}x^{a}\right)
q^{1}=0\Rightarrow \overline{F}^{\overline{q}^{2}}=-\left( m+\sqrt{2}%
ig_{a}x^{a}\right) q^{1}=-m_{eff}q^{1}.
\end{equation*}%
Calcul de $F^{x^{a}}:$
\item
\begin{equation*}
\begin{tabular}{lll}
$\frac{i}{2}\dint d^{4}\theta \left( \overline{F}_{a}X^{a}-F_{a}\overline{X}%
^{a}\right) $ & $=$ & $\frac{i}{2}\dint d^{4}\theta \left( \overline{F}_{ba}%
\overline{X}^{b}X^{a}-F_{ba}X^{b}\overline{X}^{a}\right) ,$ \\
&  & $\supset \frac{i}{2}\dint d^{4}\theta \left[
\overline{F}_{ba}\left( \overline{\theta
}^{2}\overline{F}^{\overline{x}^{b}}\right) \left( \theta
^{2}F^{x^{a}}\right) -F_{ba}\left( \theta ^{2}F^{x^{b}}\right)
\left(
\overline{\theta }^{2}\overline{F}^{\overline{x}^{a}}\right) \right] ,$ \\
&  & $=\frac{i}{2}\left[ \overline{F}_{ba}\overline{F}^{\overline{x}%
^{b}}F^{x^{a}}-F_{ba}F^{x^{b}}\overline{F}^{\overline{x}^{a}}\right] .$%
\end{tabular}%
\end{equation*}
\end{itemize}
\end{document}
